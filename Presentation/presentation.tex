\documentclass{beamer}
\usepackage{listings}
\lstset{
%language=C,
frame=single, 
breaklines=true,
columns=fullflexible
}
\usepackage{blkarray}
\usepackage{subcaption}
\usepackage{url}
\usepackage{tikz}
\usepackage{tkz-euclide} % loads  TikZ and tkz-base
%\usetkzobj{all}
\usetikzlibrary{calc,math}
\usepackage{float}
\newcommand\norm[1]{\left\lVert#1\right\rVert}
\renewcommand{\vec}[1]{\mathbf{#1}}
\usepackage[export]{adjustbox}
\usepackage[utf8]{inputenc}
\usepackage{amsmath}
\usepackage{tikz}
\usetikzlibrary{automata, positioning}
\usetheme{Boadilla}
\providecommand{\pr}[1]{\ensuremath{\Pr\left(#1\right)}}
\providecommand{\brak}[1]{\ensuremath{\left(#1\right)}}
\providecommand{\sbrak}[1]{\ensuremath{{}\left[#1\right]}}
\DeclareMathOperator*{\max}{max}
\logo{\includegraphics[height=0.8cm, center]{iith-logo}}
\title[Bayesian Estimates of Transmission Line]{Bayesian Estimates of Transmission Line Outage
Rates That Consider Line Dependencies}

\subtitle{IEEE TRANSACTIONS ON POWER SYSTEMS, VOL. 36, NO. 2, MARCH 2021}

\author[Kai Zhou, Zhaoyu Wang, Ian Dobson]
{Kai Zhou \and  James R. Cruise \and  Chris J. Dent  \and  Ian Dobson \and  Louis Wehenkel \and Zhaoyu Wang \and Amy L. Wilson }

\date[\today]
{AI5002\\ Research Paper Presentation\\ Harshal Verma (AI21MTECH02003)}



\AtBeginSection[]
{
  \begin{frame}
    \frametitle{Table of Contents}
    \tableofcontents[currentsection]
  \end{frame}
}

\begin{document}

\frame{\titlepage}
\begin{frame}{Abstract}
\begin{block}

\begin{itemize}
    \item Transmission line outages are inevitable for a power system and  the data is limited.
    \item Paper proposed a Bayesian hierarchical model that leverages line dependencies to better estimate outage rates of individual transmission lines from limited outage data.
    \item The Bayesian model produces more accurate individual line outage rates, as well as estimates of the uncertainty of these rates even when the available data is less.
    
\end{itemize}
\end{block}
\end{frame}

\begin{frame}
\frametitle{Table of Contents}
\tableofcontents
\end{frame}



\section{Introduction}

\begin{frame}
\frametitle{Introduction}
\begin{block}{}
\begin{itemize}
  \item Transmission line outage rates are foundational for
many reliability calculations, but in historical data the
counts of outages for the more reliable lines are low, and estimated individual line outage rates are highly uncertain.
  \item Ways in which individual transmission lines are partially similar, including their length, rating, geographical location, and their proximity. 
  \item This paper leverage these partial similarities with a
Bayesian hierarchical method to improve the estimation of line
outage rates from historical data.
  
\end{itemize}
\begin{flushright}
...continue
\end{flushright}

\end{block}


\end{frame}
\begin{frame}{Introduction}
\begin{block}{}
\begin{itemize}
    \item The conventional method of estimating annual line outages have high variance and for many line outages we have very less data 
    \item One method to mitigate the problem of data is to group lines by line voltages, length, area etc.
    \item To exploit the partial
dependencies of line outage rates, paper proposes a Bayesian hierarchical method to estimate outage rates of individual transmission lines. In particular, the method can leverage the multiple partial dependencies in line length, rating, network proximity,
and geographical area to give better outage rates of individual
lines. 
\end{itemize}
\begin{flushright}
...continue
\end{flushright}

\end{block}
\end{frame}



\begin{frame}{Introduction}
\begin{block}{}
The model proposes :
\begin{itemize}
    \item Estimates annual outage rates for individual transmission lines more accurately by leveraging partial similarities between line 
    \item Has performance better than the conventional method of
simply dividing the number of outages by the number of
years observed, especially when the data is limited.
    \item Instead of pooling lines with one characteristic in common, gives a way to combine multiple partial similarities
between lines. 
     
      \item  Shows that line length and rated voltage correlate with line
outage rate, but the correlation is not strong.
      
\end{itemize}

\end{block}
\end{frame}

\section{Literature review}
\begin{frame}{Literature review}
\begin{block}{}
\begin{itemize}
    \item Bayesian approaches encode uncertainty in uncertain parameters such as outage rates as random variables.
    \item Bayes theorem is used to combine
data with prior distributions that describe initial knowledge of
the uncertainty. The prior distributions are updated with the
available data to give a posterior distribution that describes the
uncertainty in the parameter values given all the available data.
   \item Bayesian methods are ideal for problems with limited data
(such as estimation of outage rates), where it is necessary to
use all the information available. 
\end{itemize}
\end{block}
\end{frame}


\section{Modelling line dependencies}
\begin{frame}{Modelling line dependencies}
\begin{block}{Historical Outage Data}
\begin{itemize}
    \item Data consist of transmission line outage by North American utility (Bonneville Power Administration) for 14 years since 1994.
    \item Data include the sending end and receiving end bus name of outaged lines, voltage rating, length, outage time, district and other information.
    \item There are 549 outages in the data with rated voltage of 69,115,230,287,345 and 500KV
    \item Excluded 1000KV HVDC line and momentary outages.
    \item If a line fails several times it's counted as one. 
\end{itemize}

\end{block}
   \end{frame}
   
 \begin{frame}{Modelling line dependencies}
\begin{block}{Data exploration}
\begin{itemize}
    \item Initially used the conventional method of dividing the number of outages by number of years.
    \item After pooling the overall data the mean and standard deviation comes out to be 0.6 and 0.7 respectively.
    \item The variance to mean ratio comes out to be 1.2 showing overdispersion.
    
\end{itemize}

\end{block}
   \end{frame}
   
 \begin{frame}{Modelling line dependencies}
\begin{block}{Scaling line length and voltage ratings}
 The line lengths and voltage ratings are transformed and scaled so that their magnitudes and variations are scale-free and comparable.
  
 \begin{equation}
     x_L = \frac{ln \boldsymbol{L}}{scale(ln\boldsymbol{L})} 
 \end{equation}
scale(z) = meadian(z-median(z))
 \begin{equation}
     x_V = \frac{\boldsymbol{V}/SD(\boldsymbol{V})}{scale(\boldsymbol{V}/SD(\boldsymbol{V}))} 
 \end{equation}
 It is usually considered that the line length and voltage rating have a positive correlation, but the correlation is weak: the Pearson correlation coefficient is 0.34
\end{block}
 \end{frame}  
 
 
  \begin{frame}{Modelling line dependencies}
\begin{block}{Line proximity}
\begin{itemize}
    \item The proximity of lines is quantified by the weighted sum of
two kernels.
     \item The first
kernel is based on districts. Lines in the same district are more
likely to experience the same weather conditions.
  \item The second kernel
is based on network distance in terms of line length, which, to
some extent, reflects both geographic proximity and the physical
and engineering interactions in the power grid.
\end{itemize}
\end{block}
\end{frame}    
 
  \begin{frame}{Modelling line dependencies}
\begin{block}{Line proximity - District}
There are 12 districts, and districts for each
line are represented by a feature vector $\phi_{dis} \in \{0, 1\}^{12}$ whose
coordinates correspond to the districts, and are set to 1 for each
district crossed by that line, and to 0 otherwise.
We define district kernel as :
\begin{equation}
     \Sigma_1 = exp(-{\parallel \phi_{dis}(i) - \phi_{dis}(j)\parallel_2}^2 - I_{i \neq j}) 
 \end{equation}
where $\parallel.\parallel$ stands for two norm and $I_{i \neq j}$ is an indicator function. The kernel $\Sigma_1 $ forms a correlation matrix since it is positive definite.

\end{block}
\end{frame}   
  
  
\begin{frame}{Modelling line dependencies}
\begin{block}{Line proximity - Network distance}
 The network distance between lines
$L_i$ and $L_j$ along the network lines is defined as :

$d_{ij}$ = d($L_{i}$,$L_{j}$) = minimum length in miles of a network path
joining midpoint of $L_i$ to midpoint of $L_j$ . \\
\\
The distance of line to itself is zero and the distance
of a line to a neighboring line with at least one bus in common
is half of the total length of the two lines.
\begin{equation}
     \Sigma_2 = exp[-\frac{d(L_i,L_j)}{2}] 
 \end{equation}
 \\
As d($L_i$, $L_i$)=0, the diagonal elements of $\Sigma_2$ are one.
\end{block}
\end{frame}  
 
  
\begin{frame}{Modelling line dependencies}
\begin{block}{Line proximity - Combining two kernels}
The network proximity $\Sigma$ is
the weighted sum of above two kernels:
\begin{equation}
     \Sigma = \emph{w}\Sigma_1 + (1-\emph{w})\Sigma_2 
 \end{equation}
 \\
Where $0 < w < 1$. For example, if the two kernels are equally
important, then \emph{w} = 0.5
\end{block}
\end{frame}  
 
\section{The Bayesian Hierarchical Model}  
\begin{frame}{The Bayesian Hierarchical Model}
\begin{block}{}
\begin{itemize}
    \item We assume that the outage count follows a Poisson distribution :

\begin{equation}
    \emph{N}_i \sim Poisson(\lambda_it_i), \hspace{1cm} i =1,\dots,n
\end{equation}\\

where $N_i$ is the outage count for line i over $t_i$ years,$\lambda_i$ is the annual outage rate, and n is the number of lines.

\item We assume that the outage rates $\lambda_i$ follows a Gamma distribution:
\begin{equation}
    \emph{\lambda}_i \sim Gamma(\alpha,\alpha/\mu_i), \hspace{1cm} i =1,\dots,n
\end{equation}\\
The mean outage rate $\mu_i$ is modeled via a linear regression
model with correlated lines.
\end{itemize}
\end{block}
\end{frame}

\begin{frame}{The Bayesian Hierarchical Model}
\begin{block}{}
\begin{itemize}
    \item The linear regression model assumes the predicted variable is normally distributed, but $\mu_i$
is positive and may have a large range of values, so $\mu_i$ is
transformed by a log function :

\begin{equation}
    \ln\boldsymbol{\mu} = \beta_0+\beta_L\emph{x}_L + \beta_V\emph{x}_V
\end{equation}\\

Where $\mu$, $\beta_0$ are column vector
\item $\beta_0$ follows a multi variable normal distribution
\begin{equation}
    \beta_0 \sim \mathcal{N}(m\boldsymbol{1},\sigma^2(\emph{w}\Sigma_1 + (1-\emph{w})\Sigma_2))
\end{equation}\\
\item Line dependencies are captured by the covariance matrix of this
multivariate normal distribution, $\sigma^2$ is a scalar which controls
the magnitude of the covariance and w controls the weights of
the two kernels.
\end{itemize}
\end{block}
\end{frame}


\begin{frame}{The Bayesian Hierarchical Model}
\begin{block}{}
\begin{itemize}
    \item The parameters $\alpha$, $\beta_V$ , $\beta_L$ , $\emps{w}$, $\sigma^2$ and $\emps{m}$ will be
estimated using prior distributions in combination with the data
as described below.
The prior distributions are:


\begin{equation}
\begin{split}
\alpha &\sim \text{Half Normal}(0.7,8^2) \\
\emph{m} &\sim Normal(-1.5,5^2)\\
\sigma^2 &\sim \text{Half Normal}(0,0.5^2)\\
\beta_L &\sim Normal(0.13,5^2)\\
\beta_V &\sim Normal(0.12,5^2)\\
\emph{w} &\sim Beta(1,1)
\end{split}   
\end{equation}

\item These priors are set to ensure that the parameters have a reasonable range and/or mean when compared to our knowledge
about the system.
\end{itemize}
\end{block}
\end{frame}

\begin{frame}{The Bayesian Hierarchical Model}
\begin{block}{Summary of Bayesian model}
\begin{itemize}
    \item We now summarize the Bayesian hierarchical model. The
Bayesian hierarchical model is specified by (6,7,8,9) together with the prior distribution of the parameters (10). Note
that the partial dependencies between the lines are expressed in
(8,9).
 \item The model parameters, including the outage rates $\lambda$, are :
\begin{equation}
  \theta = (\lambda, \mu, \beta_0, \alpha, \beta_L, \beta_V ,\emps{m}, \emps{w} )
\end{equation}


\end{itemize}
\end{block}
\end{frame}


\begin{frame}{The Bayesian Hierarchical Model}
\begin{block}{Summary of Bayesian model}
\begin{itemize}
    \item The objective is to estimate the posterior distribution of the
parameters $\pr{\theta|\boldsymbol{\emps{N}}}$ that is informed by the line outage counts
$\boldsymbol{\emps{N}}$.
 \item  By Bayes’ theorem, the posterior distribution :
\begin{equation}
  \pr{\theta|\boldsymbol{\emps{N}}} = \frac{\pr{\boldsymbol{\emps{N}}|\theta}\times \pr{\theta}}{\pr{\boldsymbol{\emps{N}}}}
\end{equation}

\end{itemize}
\end{block}
\end{frame}

\begin{frame}{The Bayesian Hierarchical Model}
\begin{block}{Summary of Bayesian model}
\begin{itemize}
    \item Normalization can be applied later, it is sufficient to
calculate the unnormalized numerator of (12). We can exploit
the dependencies in the hierarchical model (7,8,9) to get
parameters : 
\begin{equation}
 \pr{\boldsymbol{\emps{N}}|\theta} = \pr{\boldsymbol{\emps{N}}|\lambda} = \prod_{i} \pr{\emps{N}_i|\lambda_i}
\end{equation}

\begin{equation}
\resizebox{0.9\hsize}{!}{
    \pr{\theta} = \prod_{i} \pr{\lambda_i|\alpha,\mu_i}\pr{\boldsymbol{\mu}|\beta_0,\beta_L,\beta_V}\pr{\beta_0|m,w}\times \pr{\alpha}\pr{\beta_L}\pr{\beta_V}\pr{m}\pr{w}}
\end{equation} 
so that 

\begin{equation}
    \pr{\boldsymbol{\theta}|\boldsymbol{N}} \propto \pr{\boldsymbol{N}|\theta}\pr{\theta}\\
    \propto \prod_{i} \pr{\emps{N}_i|\lambda_i} \prod_{i} \pr{\lambda_i|\alpha,\mu_i}\pr{\boldsymbol{\mu}|\beta_0,\beta_L,\beta_V}\pr{\beta_0|m,w}\times \pr{\alpha}\pr{\beta_L}\pr{\beta_V}\pr{m}\pr{w}
\end{equation}
\end{itemize}
\end{block}
\end{frame}
\section{Bayesian processing on real data}
\begin{frame}{Bayesian processing on real data}
\begin{block}{Sampling Process}
The Bayesian hierarchical model described in the previous
section is applied to the historical outage data
\begin{itemize}
    \item The posterior distributions (15) of the parameters (11) can
be evaluated numerically by repeated sampling from the distribution with a Monte Carlo Markov Chain (MCMC) algorithm.
\item Software Stan was used, which implements MCMC as Hamiltonian Monte Carlo (HMC) 

\end{itemize}
\end{block}
\end{frame}


\begin{frame}{Bayesian processing on real data}
\begin{block}{Results of Bayesian estimates}
\begin{itemize}
  \item
The posterior distributions of βL and βV and their correlation, we see a very weak correlation which is reasonable as $x_L$ and $x_V$ have very weak correlation.
\end{itemize}
\end{block}
\begin{figure}
    \centering
    \subfloat[Distribution of $\beta_L$ and $\beta_V$]{{\includegraphics[width=5cm]{har.png} }}
    \qquad
    \subfloat[Scatter plot and co relation of $\beta_L$ and $\beta_V$]{{\includegraphics[width=5cm]{har1.png} }}
    \label{fig:example}
\end{figure}
\end{frame}


\begin{frame}{Bayesian processing on real data}
\begin{block}{Results of Bayesian estimates}
\begin{itemize}
  \item The conventional method estimates the
outage rate with the sample mean. The standard deviation of the
sample mean can be estimated as s/$\sqrt{n}$, where s is the sample
standard deviation, and n is the sample size.
\item The standard
deviation of the Bayesian estimator is typically smaller than the
conventional estimator, especially when the data is limited to
one year. 
\item The median ratio of standard deviations is 0.66 for one
year of data, while the median ratio is 0.93 for 14 years of data.
\item the Bayes estimator using one year of data achieves the
same standard deviation as the conventional estimator using
2.30 years of data (1/(0.662)=2.30). 
\item One advantage of the Bayesian method is that it provides a
principled way of making line outage rates with no observed
outages.
\end{itemize}
\end{block}
\end{frame}

\section{Test Bayesian estimate on synthetic data}
\begin{frame}{Test Bayesian estimate on synthetic data}
\begin{block}{}
\begin{itemize}
  \item Build a generative model for synthetic datasets of arbitrary
size, so the data are not limited in size, and the ground truth
values are known.
\item  Then test the Bayesian hierarchical model
and the conventional estimates on the synthetic data.
\item  Generate three datasets with different sizes so that we have
the equivalents of 1-year, 5-year, and 100-year data. 
\end{itemize}
\end{block}
\end{frame}


\begin{frame}{Test Bayesian estimate on synthetic data}
\begin{block}{}
\begin{itemize}
  \item The less the data the wider the histogram, as the data increases the two modes merges into one.
  \item For 1-year data, the standard deviation of the error is 0.6 for
Bayesian estimates and 0.9 for conventional estimates; for 5-year
data, the standard deviation is 0.3 for Bayesian estimates and 0.4
for conventional estimates.
\end{itemize}
\begin{figure}
    \centering
    \subfloat[Distributions of point estimation errors of Bayes estimates and conventional estimates using 1-year data]{{\includegraphics[width=5cm]{her.png} }}
    \qquad
    \subfloat[Distributions of point estimation errors of Bayes estimates and conventional estimates using 5-year data]{{\includegraphics[width=5cm]{her1.png} }}
    \label{fig:example}
\end{figure}
\end{block}
\end{frame}

\section{ Conclusion and Improvements}
\begin{frame}{Conclusion}
\begin{block}{}
\begin{itemize}


\item We use a Bayesian hierarchical model to improve the estimation of annual outage rates for individual transmission lines.
\item This
Bayesian method incorporates several types of dependencies
between lines and is applied to real outage data and tested with
synthetic data.

\item  For the shorter observation periods
with the lower outage counts, the Bayesian estimates perform
better than the conventional estimates.

\item Our Bayesian hierarchical model offers an improvement over
the conventional estimates for two reasons.  
\begin{itemize}
    \item The Bayesian
method can appropriately capture our prior knowledge of the parameter uncertainties with prior distributions.

  \item The model is hierarchical and models the dependence between
lines, information about multiple partial commonalities can be
appropriately shared across similar lines.
\end{itemize}
\end{itemize}
\end{block}
\end{frame}






\begin{frame}{Conclusion}
\begin{block}{}
\begin{itemize}
  \item For correlated line with respect to the geographical dependencies we model these line dependencies as a covariance matrix in
the Bayesian hierarchical model.
\item The covariance matrix is the
weighted sum of two kernels that represent geographic district
commonalities and network line proximity, respectively

\item The results for our data are that
individual line outage rates are only partially correlated with the
line length or the voltage rating. Therefore, it is more reasonable to consider the outage rate for a whole line instead of the rate
per mile.
\item When data is limited, which is generally true for power system
outage data, Bayesian estimates have smaller uncertainty than
conventional estimates.


\end{itemize}
\end{block}
\end{frame}


\begin{frame}{Conclusion}
\begin{block}{}
\begin{itemize}


\item With a specific acceptable
standard deviation, the proposed Bayesian method needs less
data than the conventional method.
\item For example, if utilities need two years of data
using the conventional method to estimate line outage rates with
a given uncertainty, they typically only need one year of data
using the proposed Bayesian method to obtain an outage rate
estimate that meets the same uncertainty requirement
\end{itemize}
\end{block}
\end{frame}

\begin{frame}{Improvements}
\begin{block}{}
\begin{itemize}


\item Further advantage could be gained by including other factors
such as average wind speed or altitude.
\item The proposed
Bayesian method can naturally be extended to investigate line
outage rates for specific causes.
\end{itemize}
\end{block}
\end{frame}
\end{document}
